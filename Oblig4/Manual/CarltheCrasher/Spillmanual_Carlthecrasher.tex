\documentclass[paper=a4]{article}
\usepackage{ucs}
\usepackage[utf8x]{inputenc}
\usepackage[T1]{fontenc}
\PreloadUnicodePage{0}
\usepackage{xspace}
\usepackage{array}
\usepackage[hmargin=3.5cm,vmargin=2.7cm]{geometry} 
\usepackage{graphicx,wrapfig}
\usepackage{hyperref}


\renewcommand{\contentsname}{Innhold}

\begin{document}
\title{ \normalsize \includegraphics[scale=0.43]{images/CarGameLogo.png}
	\\
	\includegraphics[scale=0.5]{images/menu_car.png}
}
\author{\textbf{Spillmanual} \\
Team Dank \\
Universitetet i Bergern \\
Informatikk}
\maketitle
\newpage

\tableofcontents
\newpage

\section{Introduksjon}
Dette er et spill tiltenkt datamaskiner der du som spiller er Carl og kjører en bil. 
Som Carl liker du å kjøre uforsvarlig, og målet ditt er å komme deg så lengst mulig uten gå tom for bensin. 
Bilen kjører langs en vei med forskjellige ting i veibanen, og som Carl ønsker du å kjøre på alt som gir deg lykke. 
På veien er det også trafikanter som er i veien for Carl og den eneste måten å få de fjernet på fra veien til Carl er å kjøre på dem.
I motsetning til ting som gir Carl lykke, befinner det seg en haug med vanndammer og kumlokker som vil forhindre Carl i å kjøre langt. 
\begin{wrapfigure}{l}{2cm}
\includegraphics[width=2cm]{images/car.png}
\end{wrapfigure} 
Vanndammer vil gjøre at Carli mister drivstoff, \\ og mister litt kontrollen over bilen. \\
Kumlokk i veien gjør at Carl mister kontroll på bilen og kræsjer. \\
Bensintanker gir Carl ekstra drivstoff, slik at runden varer lenger. \\
Myke trafikanter vil gi poeng til Carl dersom de blir påkjørt. \\
Mynter kan plukkes opp slik at Carl kan kjøpe nye biler. \\
Spilleren får en score for hver spillrunde, i tillegg til at spilleren samler opp virtuelle penger som kan brukes til å bytte bil. \\
Men du må også være obs på det lokale værforholdet hos deg, fordi det påvirke veien til Carl! 

\section{Aldergrense}
Vår målgruppe for dette spillet ungdommer og voksne i alderen fra 15 til 25 år, \\ 
og fordi Carl the Crasher inneholder mild vold er aldegrensen på dette spillet 15. 
\newpage

\section{Systemkrav og installasjon}
\subsection{Systemkrav}
\begin{itemize}
	\item Ikke mindre enn 1 GB RAM minne\\
	\item Windows Vista/Linux 2.6.12/Mac OS X 10.4 eller nyere \\
	\item 50 MB ledig plass for installasjon \\
	\item Java 8 \\
	\item Tastatur, mus \\
 	\item TeamDank anbefaler deg å ha en prosessor (2005+) med \\ 
klokkefrekvens på 1.6 Ghz eller bedre.
\end{itemize}
\subsection{Installasjon}
Du må først sørge for at du har installert den nyeste versjonen av Java og \\
for å kunne spille Carl the Crasher må du laste ned $carlthecrasher.jar$-filen. \\
Deretter åpner du filen som vanlig for å starte spillet.   
\newpage

\section{Spillinstruksjoner og -regler} 
\subsection{Kontroll} 
Din primære spillinngangsenhet er tastaturet, og kan gjøre følgende: 
\begin{itemize}
	\item Du styrer Carl til høyre og venstre ved å bruke enten piltastene eller tastaturknappene \\ $A$ og $D$. 
	\item Carl kan ikke styre bilens hastighet direkte, men farten vil øke jo lenger du kjøre.
	\item Du kan ikke kjøre ut av veien, men du kan kjøre på objekter som oppstår på veien.
	\item Du kan pause spillet ved å gå til pausemenyen ved å trykke på tastaturknappen $ESC$. 
	\begin{itemize}
		\item Å trykke på $ESC$ vil gi deg muligheten til å fortsette eller avslutte spillet.
	\end{itemize}
\end{itemize}

\subsection{Objekt-interaksjon}
På veien kan det oppstå forskjellige objekter, og avhengig av objektet kan det påføre bilen en positiv eller negativ effekt. \\
{\renewcommand\labelitemi{}
\begin{itemize}
	\item \includegraphics[scale=0.2]{images/gastank.png} \textbf{Bensintank}: Kan plukkes opp for å øke drivstoffet slik at du kan kjøre lengre 
	\item \includegraphics[scale=0.5]{images/coin.png} \textbf{Mynt}: På veien kan du finne mynter som du kan plukke opp.
																Myntene kan brukes i butikken for å kjøpe nye biler.
	\item \includegraphics[scale=0.2]{images/puddle.png} \textbf{Vanndam}: Kjører du på en vanndam vil det minske mengden drivstoff du har,
																og samtidig gjøre at du vil miste litt kontrollen på bilen i en liten periode. 
	\item \includegraphics[scale=0.2]{images/manhole.png} \textbf{Kumlokk}: Dersom du kjører på et kumlokk vil du kræsje og deretter tape umiddelbart.
	\item \includegraphics[scale=0.4]{images/walker.png} \textbf{Myke trafikanter}: Å kjøre på trafikanter vil gi deg ekstra poeng som vil øke sluttpoengsummen din.
\end{itemize}
}

\subsection{Poeng}
Målet ditt er å få den høyeste poengsummen, og det finnes ulike måter å tjene opp poeng på:
\begin{itemize}
	\item Du starter alltid med 0 poeng i starten av en runde, \\men du tjener opp poeng konstant når du kjører.
	\item Poeng blir opptjent avhengig av hvor langt du kjører. 
	\item Å kjøre på myke trafikanter vil gi deg poeng, og ulike trafikanter vil gi ulik poengsum.
\end{itemize}

\subsection{Vær} 
Spillet vil tilpasse seg etter det lokale værforholdet hos deg. Om det er skyer vil det være skyer på banen,
om det regner vil det regne på banen, og om det snør vil det være snø på banen. \\
Disse dataene om værforholdene blir hentet ut fra yr.no. 
\begin{itemize} 
	\item \textbf{Sol}: Normal framgang.
	\item \textbf{Skyer}: Skyer i spillet, men værforholdet endrer ikke spillets framgang.
	\item \textbf{Regn}: Det regner i spillet og det vil dukke opp flere vanndammer	, samt flere og nye trafikanter. 
	\item \textbf{Snø}: Det er snø på bygningene og andre trafikanter vil dukke opp. Det vil også være vanskeligere å styre bilen,
 og du vil miste mye av kontrollen på bilen dersom du kjører i en vanndam.
\end{itemize}

\newpage
\section{Bilder}
	\begin{center}
	{\renewcommand\labelitemi{}
		\begin{itemize}
			\item{\makebox[13.5cm]{\includegraphics[width=1.00\textwidth]{images/main_menu.PNG}}}
			\item {\hfil Figure 1. Hovedmeny} 
			\bigskip
			\bigskip
			\bigskip
			\item{\makebox[13.5cm]{\includegraphics[width=1.00\textwidth]{images/hs_menu.PNG}}}
			\item {\hfil Figure 2. Highscoremeny}
			\bigskip
			\bigskip
			\bigskip
			\item{\makebox[13.5cm]{ \includegraphics[width=1.00\textwidth]{images/shop_menu.PNG}}}
			\item {\hfil Figure 3. Butikkmeny}
			\bigskip
			\bigskip
			\bigskip
			\item{\makebox[13.5cm]{ \includegraphics[width=1.00\textwidth]{images/credits.PNG}}}
			\item {\hfil Figure 4. Kreditering}
			\bigskip
			\bigskip
			\bigskip
			\item{\makebox[13.5cm]{ \includegraphics[width=1.00\textwidth]{images/game.PNG}}}
			\item {\hfil Figure 5. Gameplay}
			\bigskip
			\bigskip
			\bigskip
			\item{\makebox[13.5cm]{ \includegraphics[width=1.00\textwidth]{images/cloud.png}}}
			\item {\hfil Figure 6. Vær: Skyer}
			\bigskip
			\bigskip
			\bigskip
			\item{\makebox[13.5cm]{ \includegraphics[width=1.00\textwidth]{images/rain.png}}}
			\item {\hfil Figure 7. Regn}
			\bigskip
			\bigskip
			\bigskip
			\item{\makebox[13.5cm]{ \includegraphics[width=1.00\textwidth]{images/snow.png}}}
			\item {\hfil Figure 8. Snø}
			\bigskip
			\bigskip
			\bigskip
			\item{\makebox[13.5cm]{ \includegraphics[width=1.00\textwidth]{images/pause.PNG}}}
			\item {\hfil Figure 9. Pausemeny}
			\bigskip
			\bigskip
			\bigskip
			\item{\makebox[13.5cm]{ \includegraphics[width=1.00\textwidth]{images/gameover.PNG}}}
			\item {\hfil Figure 10. Game-over}
		\end{itemize}
	}
	\end{center} 

\newpage
\section{Credits}
\begin{center} 
\textbf{TeamDank}\\ \

\textbf{Utvikler team:}\\
TeamDank Programming Team\\ \

\textbf{Grafikk:} \\
TeamDank Art Department \\ \


\textbf{Værdata:} \\ 
Weather forecast from Yr, delivered by the Norwegian Meteorological Institute and NRK. \\ 
http://om.yr.no/verdata/xml/  \\
(Data from http://www.yr.no/sted/Hordaland/Bergen/Bergen/) \\ \

\textbf{Lyder:} \\ 
Lyder brukt fra freesound: \\
car\textunderscore drive.wav (In Car Driving.aif) by (http://freesound.org/people/RutgerMuller/) \\
happy\textunderscore bgmusic.wav (Happy 8bit Loop 01) by (http://freesound.org/people/Tristan\textunderscore Lohengr) \\
car\textunderscore crash.mp3 (Car Crash) by (http://freesound.org/people/squareal) \\
coin\textunderscore sound.wav (Coins 1) by (http://freesound.org/people/ProjectsU012) \\
dead\textunderscore pedestrian (me\textunderscore ouch.wav) by (http://freesound.org/people/SpazTastic) \\
fuel.wav (Drip2.wav) by (http://freesound.org/people/Neotone) \\
splash.mp3 (Splash 1.wav) by (http://freesound.org/people/FreqMan) \\ \

\textbf{Utviklere}: \\
Peter Andre Johansen \\
Håvar Eggereide \\
Kenneth Apeland \\
Philip Thao Hoang \\
Markus Johan Ragnhildstveit \\
Torbjørn Ola Sunnarvik Moen \\
Elias Refvem Siljan \\
Sturle Fraga Elvestad \\
Eirik Strøm \\
Amund Lindberg \\
Ole Magnus Lie \\
Bjørnar Herland \\ \

\textbf{Grafikk:} \\
Malin Øien \\
Emilia Botnen Van den Bergh \\ \

\textbf{Coach:} \\
Gunnar Schulze \\ \ 

\textbf{Juridisk Team:} \\ 
Anna Fossen-Helle \\ \

\textbf{Spillidé:} \\
$Team 1\textunderscore9$ \\
Kenneth Apeland \\
Natalie Priyaphon Wannaphong \\
Joakim Moss Grutle \\
Peder Ben \\ \

\section{Copyright}

 Copyright (C) 2017  TeamDank\\

 This program is free software: you can redistribute it and/or modify \\
 it under the terms of the GNU General Public License as published by \\
 the Free Software Foundation, either version 3 of the License, or \\
 (at your option) any later version.\\ \
 
 This program is distributed in the hope that it will be useful, \\
 but WITHOUT ANY WARRANTY; without even the implied warranty of \\
 MERCHANTABILITY or FITNESS FOR A PARTICULAR PURPOSE.  See the \\
 GNU General Public License for more details.\\ \

You have received a copy of the GNU General Public License \\
along with this program.  If not, see <http://www.gnu.org/licenses/>. \\ \

Too see the full license see: \\
Lisens for TeamDank
%%\href{https://gitlab.uib.no/inf112-v2017/inf112-v2017-pr2_team_d/raw/9c544e639247bd83eb9e005f4c7615680b1d9439/LICENSE}{Full license}
\end{center}

\end{document}